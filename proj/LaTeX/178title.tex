\begin{titlepage}
\begin{center}

% Upper part of the page. The '~' is needed because \\
% only works if a paragraph has started.
%\includegraphics[width=0.15\textwidth]{./logo}~\\[1cm]

\textsc{\LARGE CSU, Sacramento}\\[1.5cm]

\textsc{\Large EEE 178: Machine Vision}\\[0.5cm]

% Title
\HRule \\[0.4cm]
{ \huge \bfseries 3D Object Tracking Using Stereo Vision\\[0.4cm] }

\HRule \\[1.5cm]

% Author and supervisor
\begin{minipage}{0.4\textwidth}
\begin{flushleft} \large
\emph{Authors:}\\
Curtis \textsc{Muntz}\\
David \textsc{Larribas}\\
Thao \textsc{Chau} \\
\end{flushleft}
\end{minipage}
\begin{minipage}{0.4\textwidth}
\begin{flushright} \large
\emph{Instructor:} \\
Dr. \textsc{Belkhouche}\\
\end{flushright}
\end{minipage}
\\[2.5cm]
\begin{abstract}
A system for offline 3D tracking using stereo vision is implemented in MATLAB. Calibration of a stereo rig is explored. Various methods must be implemented to isolate and track an object. An orange ball is used as the object to simplify the isolation techniques in order to prove functionality. 
\end{abstract}

\smallskip
\noindent \textbf{Keywords.} Object Tracking, Camera Calibration, Stereo Vision
\vfill
% Bottom of the page
{\large \today}

\end{center}
\end{titlepage}