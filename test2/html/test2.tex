
% This LaTeX was auto-generated from an M-file by MATLAB.
% To make changes, update the M-file and republish this document.

\documentclass{article}
\usepackage{graphicx}
\usepackage{color}

\sloppy
\definecolor{lightgray}{gray}{0.5}
\setlength{\parindent}{0pt}

\begin{document}

    
    
\subsection*{Contents}

\begin{itemize}
\setlength{\itemsep}{-1ex}
   \item Problem 1
   \item Part 2: Transform m's Into Pixels
   \item Part 3: Transform m's into squares's
   \item Part 4: Discussion of Results
   \item Problem 2
   \item Part 3: Minimum Distance Classifier
   \item Part 4: Perceptron
   \item Part 5 Support Vector Machine
   \item Part 6 Discussion of Results
\end{itemize}
\begin{verbatim}
% pre-processing
clear all, close all, clc;
I1=imread('exam2letters.jpg');
I2=I1;
I1=rgb2gray(I1);
figure; imshow(I1);
\end{verbatim}

        \color{lightgray} \begin{verbatim}Warning: Image is too big to fit on screen;
displaying at 67% 
\end{verbatim} \color{black}
    
\includegraphics [width=4in]{test2_01.eps}


\subsection*{Problem 1}

\begin{par}
I was unable to use optimal thresholding due to the fact that the image contains yellow text, which falls below the Otsu's thresholding value. To compensate, a higher threshold was used. Unfortunately,using such a high threshold grabs a lot of noise from the image, and even combines some letter pairs into single objects. I worked around this issue by selecting one of the $m$'s from the yellow text, and evaluating its regionprops. Because the $m$ chosen was smaller in size than the majority of the other $m$'s, I set a testing range for the regionprops. The values of the testing regionprops had to be within $90\% < m < 140\%$. Satisfying these conditions, there were 51 $m$'s found in the image.
\end{par} \vspace{1em}
\begin{verbatim}
clc; close all;
clearvars -except I1 I2 I3; close all
%thresh=graythresh(I1);
thresh=0.83;
I = im2bw(I1,thresh); %convert to bw
I = imcomplement(I); %objects are white in matlab

%I=imerode(I,B);
%I=imclose(I,B);
imshow(I);

%example M
%bigM=imcrop(I, [173,47,(195-173),(63-47)]);
smallM=imcrop(I, [525,495,(546-525),(511-495)]);
theM=smallM;
figure('name','the m'), imshow(theM), title('the m');
mSTATS=regionprops(theM,'all');

reconstructed = false(size(I));
stats = regionprops(I,'all');
figure('name', 'ms are highlighted');
imshow(I), title('ms are highlighted'); hold on;

lowerBound=0.9;
upperBound=1.4;
for i=1:size(stats)
    if ( (stats(i).Solidity < (mSTATS.Solidity*upperBound)) ...
            && (stats(i).Solidity > (mSTATS.Solidity*lowerBound)) && ...
            (stats(i).FilledArea < (mSTATS.FilledArea*upperBound)) && ...
            (stats(i).FilledArea > (mSTATS.FilledArea*lowerBound)) && ...
            (stats(i).Area < (mSTATS.Area*upperBound)) && ...
            (stats(i).Area > (mSTATS.Area*lowerBound)) && ...
            (stats(i).EulerNumber == 1) && ...
            (stats(i).ConvexArea < (mSTATS.ConvexArea*upperBound)) && ...
            (stats(i).ConvexArea > (mSTATS.ConvexArea*lowerBound)) && ...
            (stats(i).Perimeter > (mSTATS.Perimeter*lowerBound)) && ...
            (stats(i).Perimeter < (mSTATS.Perimeter*upperBound)) && ...
            (stats(i).Extent > (mSTATS.Extent*lowerBound)) && ...
            (stats(i).Extent < (mSTATS.Extent*upperBound)) && ...
            (stats(i).MajorAxisLength < (mSTATS.MajorAxisLength*upperBound)) && ...
            (stats(i).MajorAxisLength > (mSTATS.MajorAxisLength*lowerBound)) && ...
            (stats(i).MinorAxisLength < (mSTATS.MinorAxisLength*upperBound)) && ...
            (stats(i).MinorAxisLength > (mSTATS.MinorAxisLength*lowerBound)) && ...
            (stats(i).Eccentricity < (mSTATS.Eccentricity*upperBound)) && ...
            (stats(i).Eccentricity > (mSTATS.Eccentricity*lowerBound)))
        N= floor(stats(i).Centroid(1));
        M= floor(stats(i).Centroid(2));
        reconstructed(M,N) = true;
        plot(N,M,'s', 'color','red')
    end
end
hold off;
\end{verbatim}

        \color{lightgray} \begin{verbatim}Warning: Image is too big to fit on screen;
displaying at 67% 
Warning: Image is too big to fit on screen;
displaying at 67% 
\end{verbatim} \color{black}
    
\includegraphics [width=4in]{test2_02.eps}

\includegraphics [width=4in]{test2_03.eps}

\includegraphics [width=4in]{test2_04.eps}


\subsection*{Part 2: Transform m's Into Pixels}

\begin{par}
To transform the m's into single pixels, we can simply run a hit and miss operation using our m mask and the complement of the m mask. But because I elected to use region properties for this example, it was relatively easy to just include a Centroid calculation and toggle the centroid of the detected m into a true pixel value. This code can be seen in the previous section's for loop. The total number of m's in the image was 51.
\end{par} \vspace{1em}
\begin{verbatim}
figure('name','m are now pixels'), imshow(reconstructed), title('m now pixels')
mCount = sum(sum(reconstructed))
\end{verbatim}

        \color{lightgray} \begin{verbatim}Warning: Image is too big to fit on screen;
displaying at 67% 

mCount =

    51

\end{verbatim} \color{black}
    
\includegraphics [width=4in]{test2_05.eps}


\subsection*{Part 3: Transform m's into squares's}

\begin{par}
I chose to transform the m's in the image into squares. By exploiting the funcitonality of dilation, I dilate the pixel image with a square, and add the resultant image to the original image.
\end{par} \vspace{1em}
\begin{verbatim}
remover=zeros(size(theM));
remover=imcomplement(remover);
Img = imdilate(reconstructed,remover);

square = strel('square',15);
adder = imdilate(reconstructed, square);
figure('name','adder'), imshow(adder)
Img2=floor(I+Img); %figure('name','img'), imshow(Img2);
Im3=Img2+adder;


figure('name','m are now squares'), imshow(Im3), title('m now different')
\end{verbatim}

        \color{lightgray} \begin{verbatim}Warning: Image is too big to fit on screen;
displaying at 67% 
Warning: Image is too big to fit on screen;
displaying at 67% 
\end{verbatim} \color{black}
    
\includegraphics [width=4in]{test2_06.eps}

\includegraphics [width=4in]{test2_07.eps}


\subsection*{Part 4: Discussion of Results}

\begin{par}
After counting the $m$'s individually, I have verified that they were all found using my method.
\end{par} \vspace{1em}


\subsection*{Problem 2}

\begin{par}
In order to classify these two classes, they are first plotted to determine if they are seperable. The plot reveals that they are linearly seperable by a plane.
\end{par} \vspace{1em}
\begin{verbatim}
clear all; clc; close all;
class1=[0.8963, 1.4780, 0.9023;
        1.1023, 0.7144, 0.8438;
        0.7078, 1.0981, 1.3255;
        0.8516, 1.1091, 0.8739;
        1.0255, 1.2991, 1.0301;
        0.7408, 0.7857, 0.9575;
        0.9521, 0.7835, 1.0452;
        1.3166, 0.9372, 0.9489;
        0.4496, 0.8064, 0.6517;
        0.9034, 1.1314, 1.3808;]

class2=[3.7985, 1.8831, -0.1202;
        2.8175, 2.4902, 0.9653;
        2.7783, 0.5197, 0.7404;
        1.9085, 1.2397, -0.4936;
        3.2815, 1.1903, -1.0368;
        2.2147, 4.0108, 0.2256;
        2.3083, 1.0618, 1.8742;
        2.1250, 2.5301, -0.7521;
        2.8540, 2.3891, -0.9560;
        2.0397, 1.5494, 0.3092;]
figure('name','3dplot')
plot3(class1(:,1), class1(:,2), class1(:,3), 'p','color','blue'), hold on
plot3(class2(:,1), class2(:,2), class2(:,3), 's','color','red'), hold on
xlabel('feature 1')
ylabel('feature 2')
zlabel('feature 3')
title('3D plot of features');
legend('Class1','Class2'), hold off
\end{verbatim}

        \color{lightgray} \begin{verbatim}
class1 =

    0.8963    1.4780    0.9023
    1.1023    0.7144    0.8438
    0.7078    1.0981    1.3255
    0.8516    1.1091    0.8739
    1.0255    1.2991    1.0301
    0.7408    0.7857    0.9575
    0.9521    0.7835    1.0452
    1.3166    0.9372    0.9489
    0.4496    0.8064    0.6517
    0.9034    1.1314    1.3808


class2 =

    3.7985    1.8831   -0.1202
    2.8175    2.4902    0.9653
    2.7783    0.5197    0.7404
    1.9085    1.2397   -0.4936
    3.2815    1.1903   -1.0368
    2.2147    4.0108    0.2256
    2.3083    1.0618    1.8742
    2.1250    2.5301   -0.7521
    2.8540    2.3891   -0.9560
    2.0397    1.5494    0.3092

\end{verbatim} \color{black}
    
\includegraphics [width=4in]{test2_08.eps}


\subsection*{Part 3: Minimum Distance Classifier}

\begin{par}
Using the equations defined in class, the following equation was derived in order to find the Minimum Distance Classifier.
\end{par} \vspace{1em}
\begin{verbatim}
%$Z=\frac{(-2*X*\bar{X_{1}}-(\bar{X_{1}}^2)+(2*Y*\bar{Y_{1}})-(\bar{Y_{1}}^2)-(\bar{Z_{1}}^2)-(2*X*\bar{X_{2}})+(\bar{X_{2}}^2)-(2*Y*\bar{Y_{2}})+(\bar{Y_{2}}^2)+(\bar{Z_{2}}^2)}{2*(\bar{Z_{2}}-\bar{Z_{1}})}$
clc; close all;

%MDC C:
x1b=mean(class1(1,:));
x2b=mean(class2(1,:));

y1b=mean(class1(2,:));
y2b=mean(class2(2,:));

z1b=mean(class1(3,:));
z2b=mean(class2(3,:));

X=0:5;
Y=0:5;

Z=zeros(6,6);
for i=0:5
    for j=0:5
        Z(i+1,j+1)=((2*i*x1b)-(x1b^2)+(2*j*y1b)-(y1b^2)-(z1b^2)-(2*i*x2b)+(x2b^2)-(2*j*y2b)+(y2b^2)+(z2b^2))/(2*(z2b-z1b));
    end
end




figure('name','MDC wooooo');
%
%surf(X,Y,Z), hold on
plot3(class1(:,1), class1(:,2), class1(:,3), 's','color','red'), hold on
plot3(class2(:,1), class2(:,2), class2(:,3), 's','color','blue'), hold on
surf(X, Y, Z), colormap([0,1,0]);
%axis([0, 2, 0, 2, 0, 2]), title('MDC wooooooo');
xlabel('feature1'); ylabel('feature2'); zlabel('feature3');
title('MDC seperator'); legend('Class 1', 'Class 2', 'MDC'), hold off;
\end{verbatim}

\includegraphics [width=4in]{test2_09.eps}


\subsection*{Part 4: Perceptron}

\begin{verbatim}
clc;
clear Z
%build training set
trainingSet = [class1;class2]
[m, n] = size(trainingSet);
resultSet=ones(1,m)';
resultSet(11:20)=0;
weightVector = ones(1,n)/5; %

threshold = 0;%threshold to decide if the output is good or bad. usually this is 0
error_count = 1;
bias = 0.1;
iterationNo = 1;
learningRate = 1;
% training phase
while (error_count > 0)
    error_count = 0;
    for i=1:m
        gx=dot(weightVector,trainingSet(i,:))+bias;
            if (gx > threshold)
                result = 1;
            else
                result = 0;
            end

            error = resultSet(i)-result;
            if (error ~= 0)
                error_count = error_count + 1;
                weightVector = weightVector + (learningRate*(error))*trainingSet(i,1:n);
                bias = bias + learningRate*error;
            end
    end

    if (iterationNo >= 100000)
        disp('Neuron input calculation couldn''t completed in timely fashion.');
        break
    end
    iterationNo = iterationNo +1;
end
disp(['answer converged in ' num2str(iterationNo) ' iterations']);
disp(['weights: ' num2str(weightVector)]);
disp(['bias: ' num2str(bias)]);


X=0:5;
Y=0:5;
Z=zeros(6,6);
for i=0:5
    for j=0:5
        Z(i+1,j+1)=(-(weightVector(1)*i)-(weightVector(2)*j)-bias)/weightVector(3);
    end
end



figure('name','Now with NN!');
plot3(class1(:,1), class1(:,2), class1(:,3), 's','color','red'), hold on
plot3(class2(:,1), class2(:,2), class2(:,3), 's','color','blue'), hold on
surf(X,Y,Z), colormap([0,1,0]), hold on
xlabel('feature1'); ylabel('feature2'); zlabel('feature3');
legend('Class 1', 'Class 2', 'Perceptron'), title('Perceptron Solution'), hold off;
\end{verbatim}

        \color{lightgray} \begin{verbatim}
trainingSet =

    0.8963    1.4780    0.9023
    1.1023    0.7144    0.8438
    0.7078    1.0981    1.3255
    0.8516    1.1091    0.8739
    1.0255    1.2991    1.0301
    0.7408    0.7857    0.9575
    0.9521    0.7835    1.0452
    1.3166    0.9372    0.9489
    0.4496    0.8064    0.6517
    0.9034    1.1314    1.3808
    3.7985    1.8831   -0.1202
    2.8175    2.4902    0.9653
    2.7783    0.5197    0.7404
    1.9085    1.2397   -0.4936
    3.2815    1.1903   -1.0368
    2.2147    4.0108    0.2256
    2.3083    1.0618    1.8742
    2.1250    2.5301   -0.7521
    2.8540    2.3891   -0.9560
    2.0397    1.5494    0.3092

answer converged in 7 iterations
weights: -3.8326     -0.9679      3.5008
bias: 3.1
\end{verbatim} \color{black}
    
\includegraphics [width=4in]{test2_10.eps}


\subsection*{Part 5 Support Vector Machine}

\begin{par}
The equations derived in class were modified to produce the optimization problem. First, the data points were inspected and the points that were closest to the other groups were chosen. Next, a $Q$ function was solved in terms of $\alpha$ and dot product of the closest classes. Setting the differential of this $Q=0$, $\alpha$ was solved. This was plugged inot the equations to produce a weight matrix and a bias, and the plane was plotted.
\end{par} \vspace{1em}
\begin{verbatim}
clc; close all;
clearvars -except class1 class2

x1=[1.3166, 0.9372, 0.9489]'
x2=[2.0397, 1.5494, 0.3092]'

x11=dot(x1,x1);
x12=dot(x1,x2);
x22=dot(x2,x2);

%syms a X Y Z
%Q=a+a-(1/2)*((a^2)*x11-(2*a*a*x12)+(a^2)*x22);
%alpha=double(solve(diff(Q)==0,a))

alpha = 1.5304 %FOUND BY USING THE ABOVE SYMBOLIC TOOLBOX CODE^^^^

W=(alpha*(1)*x1) + (alpha*(-1)*x2)
b=1-(alpha*(1)*x11+alpha*(-1)*x12)

%SVM_solX=(-W(2)*Y-W(3)*Z-b)/W(1);
X=0:5;
Y=0:5;

Z=zeros(5,5);
for i=0:5
    for j=0:5
        Z(i+1,j+1)=(-(W(1)*i)-(W(2)*j)-b)/W(3);
    end
end

figure('name','Now with SVM!');
plot3(class1(:,1), class1(:,2), class1(:,3), 's','color','green'), hold on
plot3(class2(:,1), class2(:,2), class2(:,3), 's','color','blue'), hold on
surf(X,Y,Z), colormap([1,0,0]); hold on;
xlabel('feature1'); ylabel('feature2'); zlabel('feature3');
legend('Class 1', 'Class 2', 'SVM'), title('SVM Solution'); hold off;
\end{verbatim}

        \color{lightgray} \begin{verbatim}
x1 =

    1.3166
    0.9372
    0.9489


x2 =

    2.0397
    1.5494
    0.3092


alpha =

    1.5304


W =

   -1.1066
   -0.9369
    0.9790


b =

    2.4061

\end{verbatim} \color{black}
    
\includegraphics [width=4in]{test2_11.eps}


\subsection*{Part 6 Discussion of Results}

\begin{par}
This problem was slightly harder to understand because the classes were defined to have three dimensions, so visualizing their distributions was a little more complicated than the two dimensional examples that I was used to. The seperators all worked very effectively, but it seemed that the MDC method produced the best sperator. This was surprising because the MDC seperator worked better than the SVM method, which by theory should be the optimal solution. Most of this is likely due because I drew the flat plane produced by the output of the SVM method, as opposed to the ideal kernal that could be produced through using the quadprog function.
\end{par} \vspace{1em}



\end{document}
    
