
% This LaTeX was auto-generated from an M-file by MATLAB.
% To make changes, update the M-file and republish this document.

\documentclass{article}
\usepackage{graphicx}
\usepackage{color}

\sloppy
\definecolor{lightgray}{gray}{0.5}
\setlength{\parindent}{0pt}

\begin{document}

    
    
\subsection*{Contents}

\begin{itemize}
\setlength{\itemsep}{-1ex}
   \item pre processing
   \item Problem 1: Detect, Isolate and Transform the es in an Image
   \item Part 1: Detect and Isolate e's
   \item Part 2: Transform e's Into Pixels
   \item Part 3: Transform e's into squares's
   \item Part 4: Discussion
   \item Problem 2: Hough Transform to Detect Lines in Image
   \item Problem 3:
\end{itemize}


\subsection*{pre processing}

\begin{verbatim}
clear all; close all; clc;
%code for my custom functions can be found on
%https://github.com/curtismuntz/machine_vision/tree/master/commonFunctions
addpath ../commonFunctions;
%Problem 1 Data:
I1=getIMG('letters.jpg');
%Problem 2 Data:
I2=getIMG('edges.png');
%Problem 3 Data:
I3=getIMG('test1AP1.JPG');
rmpath ../commonFunctions;
\end{verbatim}


\subsection*{Problem 1: Detect, Isolate and Transform the es in an Image}



\subsection*{Part 1: Detect and Isolate e's}

\begin{par}
In order to detect and isolate these e's, we need to define a structuring element, or set of structuring elements that will catch all of the e's. Instead of choosing the easy route and making a generic skeleton of the letter e, I elected to examine each object in the image and look at the region properties of said objects. Based on an Area threshold of $80>Area>110$, an Euler Number of $0$, and a Perimiter range of $65>Perimiter<70$, All of the e's are detected and highlighted in the following image. The original e was cropped from the image, and the coordinates were placed into matlab to simplify the manual processing.
\end{par} \vspace{1em}
\begin{verbatim}
clearvars -except I1 I2 I3; close all
thresh=graythresh(I1);
I = im2bw(I1,thresh); %convert to bw
I = imcomplement(I); %objects are white in matlab

%basic e
theE=I([28:46],[223:240],:);
figure, imshow(theE)

eSTATS=regionprops(theE,'all');

reconstructed = false(size(I));
stats = regionprops(I,'all');
figure
imshow(I), title('es are highlighted'); hold on;
for i=1:size(stats)
    if ((stats(i).Area < 110) && (stats(i).Area > 80) && (stats(i).EulerNumber == 0) && (stats(i).ConvexArea < 200) && (stats(i).ConvexArea > 150) && (stats(i).Perimeter > 65) && (stats(i).Perimeter < 70))
        N= floor(stats(i).Centroid(1));
        M= floor(stats(i).Centroid(2));
        reconstructed(M,N) = true;
        plot(N,M,'s','color','green')
    end
end
hold off;
\end{verbatim}

        \color{lightgray} \begin{verbatim}Warning: Image is too big to fit on screen;
displaying at 67% 
\end{verbatim} \color{black}
    
\includegraphics [width=4in]{hw7_01.eps}

\includegraphics [width=4in]{hw7_02.eps}


\subsection*{Part 2: Transform e's Into Pixels}

\begin{par}
To transform the e's into single pixels, we can simply run a hit and miss operation using our e mask and the complement of the e mask. But because I elected to use region properties for this example, it was relatively easy to just include a Centroid calculation and toggle the centroid of the detected e into a true pixel value. This code can be seen in the previous section's for loop. The total number of e's in the image was 84.
\end{par} \vspace{1em}
\begin{verbatim}
figure('name','e are now pixels'), imshow(reconstructed), title('e now pixels')
eCount = sum(sum(reconstructed))
\end{verbatim}

        \color{lightgray} \begin{verbatim}Warning: Image is too big to fit on screen;
displaying at 67% 

eCount =

    84

\end{verbatim} \color{black}
    
\includegraphics [width=4in]{hw7_03.eps}


\subsection*{Part 3: Transform e's into squares's}

\begin{par}
I chose to transform the e's in the image into squares. By exploiting the funcitonality of dilation, I dilate the pixel image with a square, and add the resultant image to the original image.
\end{par} \vspace{1em}
\begin{verbatim}
remover=zeros(size(theE));
remover=imcomplement(remover);
Img = imdilate(reconstructed,remover);

square = strel('square',15);
adder = imdilate(reconstructed, square);
figure('name','adder');
Img2=floor(I+Img);
Im3=Img2+adder;


figure('name','e are now squares'), imshow(Im3), title('e now different')
\end{verbatim}

        \color{lightgray} \begin{verbatim}Warning: Image is too big to fit on screen;
displaying at 67% 
\end{verbatim} \color{black}
    
\includegraphics [width=4in]{hw7_04.eps}

\includegraphics [width=4in]{hw7_05.eps}


\subsection*{Part 4: Discussion}

\begin{par}
The method of choice is explained in the previous parts.
\end{par} \vspace{1em}


\subsection*{Problem 2: Hough Transform to Detect Lines in Image}

\begin{verbatim}
clearvars -except I1 I2 I3; close all;

I=zeros(200,200); I(150,:)=1; I(151,:)=1; I(152,:)=1;
I(:,100)=1; I(:,101)=1; I(:,102)=1;
[H,T,R]=hough(I);
P  = houghpeaks(H,2);
subplot(211); imagesc(I); colormap(gray)
subplot(212); imshow(imadjust(mat2gray(H)),'XData',T,'YData',R,'InitialMagnification','fit')
title('Hough Transform');
xlabel('\theta'), ylabel('\rho');
axis on, axis normal, hold on; colormap(hot);
plot(T(P(:,2)),R(P(:,1)),'s','color','green')
\end{verbatim}

\includegraphics [width=4in]{hw7_06.eps}
\begin{par}
The peaks have been generously highlighted by green. I noticed that the houghpeaks table excludes the repeated point, $(\theta=90$, \ensuremath{\backslash}rho=150)\$ and $(\theta=-90$, \ensuremath{\backslash}rho=-150)\$.
\end{par} \vspace{1em}


\subsection*{Problem 3:}

\begin{par}
We want to find the the extraterrestrial beings (aliens) in the image. I used matlab's findcircles function to determine the where circles exist, and then decided based on those answers which circles resemble the alien's eyes.
\end{par} \vspace{1em}
\begin{verbatim}
clearvars -except I1 I2 I3; close all;
thresh=graythresh(I3);
I=im2bw(I3,thresh);

[centers, radii, metric] = imfindcircles(I3,[5 20]);
centersStrong1 = centers(1,:);
radiiStrong1 = radii(1);
metricStrong1 = metric(1);

centersStrong2 = centers(5,:);
radiiStrong2 = radii(5);
metricStrong2 = metric(5);

imshow(I3), hold on;
viscircles(centersStrong1, radiiStrong1,'EdgeColor','r'), hold off;
viscircles(centersStrong2, radiiStrong2,'EdgeColor','r'), hold off;
\end{verbatim}

        \color{lightgray} \begin{verbatim}Warning: You just called IMFINDCIRCLES with a large
radius range. Large radius ranges reduce algorithm
accuracy and increase computational time. For high
accuracy, relatively small radius range should be
used. A good rule of thumb is to choose the radius
range such that Rmax < 3*Rmin and (Rmax - Rmin) <
100. If you have a large radius range, say [20 100],
consider breaking it up into multiple sets and call
IMFINDCIRCLES for each set separately, like this:

	[CENTERS1, RADII1, METRIC1] =
        IMFINDCIRCLES(A, [20 60]);
	[CENTERS2, RADII2, METRIC2] =
        IMFINDCIRCLES(A, [61 100]);
 
Warning: You just called IMFINDCIRCLES with very
small radius value(s). Algorithm accuracy is limited
for radius values less than 10.
 
Warning: Image is too big to fit on screen;
displaying at 67% 

ans =

  178.0051


ans =

  181.0051

\end{verbatim} \color{black}
    
\includegraphics [width=4in]{hw7_07.eps}



\end{document}
    
